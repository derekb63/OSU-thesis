%!TEX root = thesis.tex

\chapter{Conclusions \& Future Work}
\label{part:conclusion}
\label{part:results}
The preceding chapters of this document presented findings from studies evaluating the influence of various parameters on the probability of ignition of fuel beds in wild fires. These studies were conducted to address the specific objectives of this work and to add to the current body of knowledge regarding ignition with the overall goal of facilitating a model for quickly and accurately predicting ignition risk to reduce the number of homes lost to wildfires. The specific objectives of this work are as follows:
        \begin{enumerate}
            \item Determine the effects of particle morphology on ignition propensity
            \item Ascertain ignition dependence on heating location(s), mode, and rate
            \item Identify and quantify the influence of environmental conditions on ignition propensity.
            \item Identify the influence of fuel bed chemical composition on flaming ignition.
        \end{enumerate}
The conclusions of each effort are restated in the following sections followed by a discussion of the implications of the results as they pertain to each specific objective and the overall objective of this work.

\section{Particle Size}
        \begin{enumerate}
            \item 
                Smaller particles ignite more readily in porous beds than larger particles when heat transfer from the heater is primarily through conduction. This was evident by higher ignition probabilities, in general, of the smaller particles for a fixed heater temperature. As particle sizes increase radiant heat transfer becomes more important and fuel beds with larger particles were more likely than smaller particles to ignite at extended times (\textgreater 100\si{\second}) due to the increased importance of radiant ignition. 
            \item
                Douglas-fir plates ignite at times where conduction is the dominant mode of heat transfer (\textless 10\si{\second}) due to the higher thermal conductivity of the solid plates. The ignition probability of plates was the most similar to the larger particle, in particular at lower heater temperatures, due to dispersed heating of the porous fuel bed through radiation and the increased thermal conductivity of the plates creating similar temperature profiles. The rise in ignition probability  over a smaller heater temperature range time with temperature results from more consistent contact between the heater and plate surface.
            \item 
                Heat flux delivered to the fuel bed, when compared to heater temperature, is more indicative of ignition likelihood and ignition time for porous fuel beds. Heat flux is a more significant predictor of ignition because it captures differences in heat transfer modes and particle contact that heater temperature values do not. While this finding is not new, what is novel is that the mixed mode of heating (conduction and radiation) has a significant impact on the flaming ignition of fuel beds.
            \item 
                Consideration of the transport characteristics of pyrolyzate gases near the high temperature source can be important for more fully predicting ignition propensity. A \textit{Da} of ignition, in relation to the measured heat flux and thermal diffusivity of the fuel beds, is a promising relationship for predicting ignition for the porous fuel beds.  
        \end{enumerate}

\section{Wind Speed}
        \begin{enumerate}
            \item
            An increase in wind speed above quiescent conditions reduces the temperature required for the flaming ignition of a fuel bed. For example, an increase in wind speed of 3.5\si{\meter\per\second} from quiescent increases the ignition probability of a fuel bed from under 30\% to roughly 100\%. However, a linear increase in wind speed does not result in a linear increase in ignition probability. Thresholds in wind speed exist above which temperatures required to achieve ignition actually increase. For example, when the wind is oriented 45\si{\degree} from the heater centerline increasing the wind from quiescent to 3.5\si{\meter\per\second} reduces the temperature required for ignition probability by 30\%. However, increasing the wind speed from quiescent to 5.8\si{\meter\per\second} reduces the temperature required for ignition by only 25\%. Presumably these thresholds occur because of reductions in residence time. 
            
            \item The temperatures at which ignition occurs for porous fuel beds is sensitive to the orientation of a firebrand relative to the wind direction. Higher temperatures are typically required for ignition for a heater parallel to the flow compared to 45\si{\degree} and perpendicular to the flow. This sensitivity attributed to differences in recirculation zones and residence times of air and pyrolyzates near the hottest region of the heater. Thus, predictions of ignition probabilities that consider wind may need to include both wind speed and orientation to obtain sufficient accuracy.
            
            \item Times to flaming ignition of porous fuel beds are sensitive to the firebrand/heater angle in the presence of wind.
            The parallel heater orientation ignites at the longest time followed by the 45\si{\degree} case with the perpendicular cases igniting in the shortest amount of time. High speed images indicate that ignition typically occurs in regions where recirculation zones occur, as shown in CFD calculations. The heightened propensity to ignition is attributed to increased residence times of pyrolyzates in the recirculation zones as supported by calculations.
        \end{enumerate}
    The conclusions of this work show that ignition is favored when a firebrand(s) land on a fuel bed under wind speeds and orientations that promote greater residence times of pyrolyzates near a high temperature region of firebrands. It was observed that increases of wind speed, of a magnitude that may commonly occur during wildfires, can increase the probability of fuel bed ignition from very unlikely to a near certainty regardless of the ember orientation to the wind. This highlights the increased risk of spot fires due to ignition of fuel beds that accompanies wind in a wildfire. 

\section{Species}

\section{Ember Interactions}
        \begin{enumerate}
            \item An increase of wind speed from 0.5\si{\meter\per\second} to 5.8\si{\meter\per\second} reduced the temperature required for ignition for all heater configurations. The decrease in ignition temperature required ranged from 20\% to 60\% depending on the heater configuration. 
            
            \item At wind speeds of 5.8\si{\meter\per\second} the ignition threshold is largely independent of the heater configuration. This lack of sensitivity is attributed to the ignition being controlled by the fluid dynamics around the firebrand surrogates. Under wind ignition is largely controlled by the propensity of pyrolysis products to accumulate in recurculation zones and independent of thermal interactions with nearby objects whether they are energy sources or sinks. 
            \item At wind speeds of 0.5\si{\meter\per\second} the ignition threshold is dependent on the firebrand surrogate configuration. For example, when the firebrand were spaced one diameter apart the difference between ignition thresholds when the upstream firebrand is inert or an energy source is 34\%. When the firebrand surrogates were spaced such that they did not thermally interact (five diameters apart) the difference in ignition threshold was 4\%. Thus, at low wind speeds or in quiescent conditions ignition is sensitive to thermal interactions with nearby objects. Ignition is promoted if nearby objects supply energy and inhibited if energy sinks are nearby. 
            
            \item In configurations where the firebrand surrogate is downstream of an inert flow obstacle ignition may occur as a smoldering to flaming transition. The smoldering to flaming transition appears to be facilitated by either accumulation of pyrolysis gases in the downstream edge of the flow obstruction or by the formation of an overhang as the fuel bed recedes underneath the obstruction. This phenomena produces ignition at similar temperatures to other configurations and may not be important for ignition predictions but may be of interest to smoldering research. 
        \end{enumerate}

\section{Overall model}
    The overall objective of this work is to identify parameters and processes that control the ignition of a fuel bed when an ember lands on it with an anticipated impact of enabling the creation of a simplified model that may be used quickly determine the likelihood of ignition. To evaluate the effectiveness of this work in meeting those goals a model was created to predict ignition. The model was then applied to results of other studies that utilized different fuel bed materials and ignition sources to determine cross study applicability.
    
    After the completion of the experimental efforts outlined in the previous chapters the results were aggregated into a single dataset containing the results from the 1086 tests conducted. The parameters included in the model and the corresponding units are as follows:
        \begin{itemize}
            \item Wind speed (\si{\meter\per\second)}
            \item Wind direction relative to longitudinal firebrand axis (\si{\degree})
            \item Ratio of characteristic lengths between particle size and ember size (-)
            \item Chemical composition: mass fractions of Cellulose, Hemicellulose, Lignin, Tannins, and Triglycerides (-)
            \item Fuel bed bulk density (\si{\kilo\gram\per\cubic\meter})
            \item Fuel bed thermal diffusivity (\si{\square\meter\per\second})
            \item Average temperature of the firebrand(s) (\si{\celsius})
            \item Number of firebrands (-)
            \item Spacing between firebrands as multiple of characteristic length (-)
        \end{itemize}
    The aggregated data was then used to create a Random Forest Classifier model using the scikit-learn package in python~\cite{scikit-learn} to estimate ignition outcomes for each test. Using a 0.25/0.75 test train spit on the data the R\textsuperscript{2} validation score for the model with an out of bag score of 0.83. Suggesting that the model accounts can predict the ignition correctly 83\% of the time. While the accuracy of the model is likely acceptable if such a model was to be used as an indicator or predictor of ignition during a fire or to inform preventative measures it would be beneficial to reduce the number of inputs required. An examination of the most important input parameters and refinement of the model found that accuracy was not degraded if number of parameters was reduced to include only the following variables:
        \begin{itemize}
            \item Average temperature of the firebrands
            \item Fuel bed bulk density
            \item Wind speed
            \item Ratio of characteristic lengths between particle size and ember size
            \item Wind direction
        \end{itemize}
    Using the five characteristics above the model, a prediction accuracy of 83\% was maintained. While it is beneficial to have an accurate predictive model for a single set of experiments in a similar apparatus, a model capable of predictions across different is far more valuable for preventing home loss. To evaluate the accuracy of the model across studies the model was applied to a selection of studies from literature. The studies and relevant parameters from each study are shown in Table~\ref{tab:otherStudies}. All of the studies presented in Table~\ref{tab:otherStudies} used hot metal particles dropped onto fine fuel beds of processed material. The studies by Hadden et al. and Urban et al. used pure cellulose as the fuel bed and the study by Zak et al used a blend of grasses processed into a powder. A wind direction of 0\si{\degree} was used for all studies since the geometry of the metal particles is anticipated to produce recirculation zones similar to that of a cylindrical heater parallel to the flow. The average temperature was defined as the average of the initial temperature and ambient temperature. More sophisticated estimates of the average temperature based on the average time to ignition and estimated heat losses from the ember were conducted but the accuracy was not improved. Thus, a simpler metric was chosen.  
        \begin{table}[]
            \centering
            \caption{Data inputs for the model from selected studies.}
            \begin{tabular}{crrrrrc}
                Authors & $\bar{T}$ (\si{\celsius}) &  $\bar{\rho}_{bed}$ (\si{\kilo\gram\per\cubic\meter})& $U$ (\si{\meter\per\second}) & $L_{p}/L_{e}$ (-)& $\Theta_{wind}$ (\si{\degree}) & Citation\\
                \hline
                 Hadden et al. & 260 - 560 & 200 & 0.5 & 0.02 - 0.45 & 0 & \cite{Hadden2011}\\
                 Urban et al. & 285 - 560 & 282 & 0.5 & 0.06 - 0.25 & 0 & \cite{Urban2018}\\
                 Zak et al.    & 310 - 560 & 338 & 0.5 & 0.04 - 0.15 & 0 & \cite{Zak2014}
            \end{tabular}
            \label{tab:otherStudies}
        \end{table}
    The first approach used was to apply reduced model created from only the data generated from studies in this dissertation and predict ignition for each of the studies in literature. The accuracy of the model at predicting ignition for other studies was 50\%, which is entirely unusable for any sort of predictive modelling. However when the model was retrained using the same five parameters but including the Zak et al. data in conjunction with the results of this work the predictive accuracy increased to 96\% for the Hadden et al. study and 81\% for the Urban et al. study. The increase in accuracy suggests that using as few as the five parameters in this model predictions can be made across differences in experimental conditions. Predictions across experimental conditions has been one of the biggest hurdles to creating a general predictive model. The identification of the primary controlling parameters of ignition in this work has created a framework for bridging the predictive gap across various configurations.  
    
\section{Future Work}
    