%!TEX root = thesis.tex

\chapter{Conclusions \& Future Work}
\label{part:conclusion}
\label{part:results}
The preceding chapters of this document presented findings from studies evaluating the influence of various parameters on the probability of ignition of fuel beds in wild fires. These studies were conducted to address the specific objectives of this work and to add to the current body of knowledge regarding ignition with the overall goal of facilitating a model for quickly and accurately predicting ignition risk to reduce the number of homes lost to wildfires. The specific objectives of this work are as follows:
        \begin{enumerate}
            \item Determine the effects of particle morphology on ignition propensity
            \item Ascertain ignition dependence on heating location(s), mode, and rate
            \item Identify and quantify the influence of environmental conditions on ignition propensity.
            \item Identify the influence of fuel bed chemical composition on flaming ignition.
        \end{enumerate}
The conclusions of each effort are restated in the following sections followed by a discussion of the implications of the results as they pertain to each specific objective and the overall objective of this work.

\section{Particle Size}
        \begin{enumerate}
            \item 
                Smaller particles ignite more readily in porous beds than larger particles when heat transfer from the heater is primarily through conduction. This was evident by higher ignition probabilities, in general, of the smaller particles for a fixed heater temperature. As particle sizes increase radiant heat transfer becomes more important and fuel beds with larger particles were more likely than smaller particles to ignite at extended times (\textgreater 100\si{\second}) due to the increased importance of radiant ignition. 
            \item
                Douglas-fir plates ignite at times where conduction is the dominant mode of heat transfer (\textless 10\si{\second}) due to the higher thermal conductivity of the solid plates. The ignition probability of plates was the most similar to the larger particle, in particular at lower heater temperatures, due to dispersed heating of the porous fuel bed through radiation and the increased thermal conductivity of the plates creating similar temperature profiles. The rise in ignition probability  over a smaller heater temperature range time with temperature results from more consistent contact between the heater and plate surface.
            \item 
                Heat flux delivered to the fuel bed, when compared to heater temperature, is more indicative of ignition likelihood and ignition time for porous fuel beds. Heat flux is a more significant predictor of ignition because it captures differences in heat transfer modes and particle contact that heater temperature values do not. While this finding is not new, what is novel is that the mixed mode of heating (conduction and radiation) has a significant impact on the flaming ignition of fuel beds.
            \item 
                Consideration of the transport characteristics of pyrolyzate gases near the high temperature source can be important for more fully predicting ignition propensity. A \textit{Da} of ignition, in relation to the measured heat flux and thermal diffusivity of the fuel beds, is a promising relationship for predicting ignition for the porous fuel beds.  
        \end{enumerate}

\section{Wind Speed}
        \begin{enumerate}
            \item
            An increase in wind speed above quiescent conditions reduces the temperature required for the flaming ignition of a fuel bed. For example, an increase in wind speed of 3.5\si{\meter\per\second} from quiescent increases the ignition probability of a fuel bed from under 30\% to roughly 100\%. However, a linear increase in wind speed does not result in a linear increase in ignition probability. Thresholds in wind speed exist above which temperatures required to achieve ignition actually increase. For example, when the wind is oriented 45\si{\degree} from the heater centerline increasing the wind from quiescent to 3.5\si{\meter\per\second} reduces the temperature required for ignition probability by 30\%. However, increasing the wind speed from quiescent to 5.8\si{\meter\per\second} reduces the temperature required for ignition by only 25\%. Presumably these thresholds occur because of reductions in residence time. 
            
            \item The temperatures at which ignition occurs for porous fuel beds is sensitive to the orientation of a firebrand relative to the wind direction. Higher temperatures are typically required for ignition for a heater parallel to the flow compared to 45\si{\degree} and perpendicular to the flow. This sensitivity attributed to differences in recirculation zones and residence times of air and pyrolyzates near the hottest region of the heater. Thus, predictions of ignition probabilities that consider wind may need to include both wind speed and orientation to obtain sufficient accuracy.
            
            \item Times to flaming ignition of porous fuel beds are sensitive to the firebrand/heater angle in the presence of wind.
            The parallel heater orientation ignites at the longest time followed by the 45\si{\degree} case with the perpendicular cases igniting in the shortest amount of time. High speed images indicate that ignition typically occurs in regions where recirculation zones occur, as shown in CFD calculations. The heightened propensity to ignition is attributed to increased residence times of pyrolyzates in the recirculation zones as supported by calculations.
        \end{enumerate}
    The conclusions of this work show that ignition is favored when a firebrand(s) land on a fuel bed under wind speeds and orientations that promote greater residence times of pyrolyzates near a high temperature region of firebrands. It was observed that increases of wind speed, of a magnitude that may commonly occur during wildfires, can increase the probability of fuel bed ignition from very unlikely to a near certainty regardless of the ember orientation to the wind. This highlights the increased risk of spot fires due to ignition of fuel beds that accompanies wind in a wildfire. 

\section{Species}

\section{Ember Interactions}

\section{Overall model}
    The overall objective of this work is to identify parameters and processes that control the ignition of a fuel bed when an ember lands on it with an anticipated impact of enabling the creation of a simplified model that may be used quickly determine the likelihood of ignition. To evaluate the effectiveness of this work in meeting those goals a model was created to predict ignition. The model was then applied to results of other studies that utilized different fuel bed materials and ignition sources to determine cross study applicability.
    
    

\section{Future Work}