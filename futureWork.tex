%!TEX root = thesis.tex

\chapter{Future Work}
\label{chap:futureWork}

  The results in this work have identified controlling parameters for ignition and provided a framework for creating models with capabilities to predict various ignition scenarios. Significant additional work is needed to refine the proposed framework and explore different ignition parameters. 
        \begin{itemize}
            \item The moisture content of the fuel bed was not considered in this work but is a well-studied parameter that significantly affects ignition. Extension of the experiments or a meta-analysis of literature is necessary to determine how moisture content fits into the proposed framework. The change in fuel bed bulk density due to moisture may sufficiently characterize the difference, but that remains to be seen.
            
            \item Additional tests with well-characterized smoldering or flaming embers would significantly enhance the knowledge gained in this work. Typical temperatures of physically combusting firebrands are higher than those capable by the apparatus used in this work~\cite{Fateev2017a}. The higher temperatures may influence the ignition propensity. However, combusting embers are susceptible to heat losses and environmental factors where the firebrand surrogates used in this work are mainly independent of these factors. It is unclear how or if the addition of energy sources that react with the environment will shift the parameters identified to be most influential to ignition. A series of highly instrumented and controlled experiments focused on the coupled interactions between a reacting ember, a reacting fuel bed, and changing environmental conditions would provide novel and valuable insight into ignition processes. I applaud the patience and perseverance of those who may undertake this endeavor in the future. 
            
            \item Relatively limited data is available regarding the thermal properties of porous biomass media. The lack of available information is likely due to the significant variation in solid biomass material properties and the endless configurations (e.g., porosity and particle orientation) in the natural environment. A series of targeted studies determining the thermal properties of common fuel bed materials would likely increase the understanding of how ignition changes between different materials or even similar materials under various packing conditions.
            
            \item In addition to the limited data available for thermal properties of fuel bed materials, the chemical composition of materials and the influence of chemical changes on the pyrolysis and subsequent ignition would provide valuable insight into the differences in ignition between materials and environmental conditions. 
        
        \end{itemize}