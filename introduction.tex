%!TEX root = thesis.tex

\chapter{Introduction}
\label{part:intro}
    Wildfires are an inevitable and often beneficial occurrence in many ecosystems around the globe. However, recent human interactions with fire prone ecosystems through forest management and fire suppression efforts have shifted the structure of such ecosystems. Specifically, human intervention often promotes a shift from frequent low intensity fires to infrequent high intensity fires~\cite{Keeley2019}. 
    However, interactions of wildfires with the built environment are often disastrous, reducing homes to rubble and uprooting lives. In recent years wildfires have become more severe and resulted in an increase in homes lost. In order to understand the cause of the increase in home loss, the mode by which homes are ignited needs addressed. One significant mode of structure ignition is by firebrands. A firebrand, alternatively called an ember, is a hot combusting particle of biomass that travels from an active fire to another surface or biomass (e.g., pine needles, leaves, or a crevice in a deck). If the firebrands have sufficient energy they  may ignite the material they land on, leading to ignition and potentially the destruction of a structure. Characterization of firebrands typically involves three parts: the generation of a firebrand in the fire itself, transport of a firebrand from the fire to the eventual landing point, and the interaction of the firebrand with the surrounding material. The overall objective of this work is to identify the parameters and processes that control the ignition of a fuel bed when an ember lands on it. Four specific objectives were identified to act as framework for achieving the overall goal based on the current knowledge of fuel bed ignition. The specific objectives of this work are as follows:
        \begin{enumerate}
            \item Determine the effects of particle morphology on ignition propensity
            \item Ascertain ignition dependence on heating location(s), mode, and rate
            \item Identify and quantify the influence of environmental conditions on ignition propensity.
            \item Identify the influence of fuel bed chemical composition on flaming ignition.
        \end{enumerate}
    The structure of this dissertation is as follows. First the current state of knowledge of fuel bed ignition is summarized as it applies to this work. The results of this effort are then presented in manuscript form, followed by the conclusions, suggestions for future work, and appendices. 
    The first manuscript addresses the effect of particle size on the ignition probability as well as the effects of heating rate resulting from changes in particle morphology. The second manuscript addresses the influence of the chemical composition on ignition propensity and how changing material properties effect the heat transfer both within the bed and between the bed and the ember. The third manuscript addresses the impact of environmental conditions, namely wind, on the ignition probability of fuel beds.
    
    Hopefully, the knowledge and conclusions of this work may be used to inform fire prevention and suppression efforts by creating more informed.
    
    This layout of this dissertation is as follows. First, the current state of knowledge with respect to fuel bed ignition, and the avenues for further research are identified. Then, the results of this work are presented in a series of four manuscripts. The manuscripts are followed by overall conclusions and suggestions for future work.
    
    The first manuscript addresses sensitivities of ignition to the particle size of the fuel bed materials. 
    
    The second manuscript explores the influence of wind and fluid phenomena around the embers in the presence of a heat source unaffected by wind.
    
    The third manuscript identifies sensitivities to ignition related to chemical composition and thermal properties in both quiescent conditions and with wind. 
    
    The fourth manuscript identifies changes in ignition propensity when multiple embers are located in close proximity on the fuel bed.