%!TEX root = thesis.tex

\chapter{Introduction}
\label{part:intro}
    Wildfires are an inevitable and often beneficial occurrence in many ecosystems around the globe. However, wildfires are also a significant source of destruction, especially when they transition from ecosystems that are adapted to them to the built environment where homes, structures, and sometimes whole communities are consumed. In recent years the number of homes and structures lost to wildfires has increased and is likely to further increase due to three factors. Climate change is currently and is anticipated to be a driving factor for increased fire severity~\cite{Levin2021Unveiling2019/2020}. Wildfire exclusion has led to significant changes in ecosystems, promoting more severe fires~\cite{Marlon2012Long-termUSA, Keeley2019} and the expansion of the Wildland Urban Interface (WUI) has increased the number of homes in the path of fires~\cite{Radeloff2018RapidRisk, Hammer2009DemographicManagement, }. While the risk of loss due to fires is likely to increase for the foreseeable future, adapting structures to the increased risk by accurately predicting risks both before and during fires may reduce the number of structures lost~\cite{Manzello2021}. 
    
   One significant mode of structure ignition is by firebrands~\cite{Manzello2020}. A firebrand, alternatively called an ember, is a hot combusting particle of biomass that travels from an active fire to another surface or biomass (e.g., pine needles, leaves, or a crevice in a deck). If the firebrands have sufficient energy, they may ignite the material they land on, leading to ignition, spot fires, and potentially the destruction of a structure. Characterization of firebrands typically involves three parts: the generation of a firebrand in the fire itself, transport of a firebrand from the fire to the eventual landing point, and the interaction of the firebrand with the surrounding material~\cite{Babrauskas2003}. 
    
    This work aims to identify parameters and processes that control the ignition of a fuel bed when an ember lands on it. Four specific objectives were addressed to act as a framework for achieving the overall goal based on the current knowledge of fuel bed ignition. The specific objectives of this work are as follows:
        \begin{enumerate}
            \item Determine the effects of sizes of the recipient fuels on ignition behavior
            \item Ascertain ignition dependence on heating location(s), mode, and rate
            \item Identify and quantify the influence of wind on ignition propensity.
            \item Identify the influence of fuel bed chemical composition on flaming ignition.
        \end{enumerate}
    The structure of this dissertation is as follows. First, the current state of knowledge of fuel bed ignition modeling is summarized as it applies to this work (Chapter~\ref{part:literature}). A literature review specific to each objective is contained in the corresponding manuscripts. The results of this effort are then presented in manuscript form, followed by the conclusions, suggestions for future work, and appendices. The first manuscript addresses sensitivities of ignition to the particle size of the fuel bed materials (Chapter~\ref{part:manuscript1}). The second manuscript explores the influence of wind and fluid phenomena around the embers in the presence of a heat source unaffected by wind (Chapter~\ref{part:manuscript2}). The third manuscript identifies sensitivities to ignition related to chemical composition and thermal properties in both quiescent conditions and with wind (Chapter~\ref{part:manuscript3}). The fourth manuscript identifies changes in ignition propensity when multiple embers are located close to the fuel bed (Chapter~\ref{part:manuscript1}). The final chapter summarizes the conclusions of each manuscript, discusses the implications of the conclusions with respect to predictive using an example ignition model, and suggests avenues for future work (Chapter~\ref{part:conclusion}). 
    
    It is anticipated that this work will provide insights and a basis for the creation of a simplified model that can accurately predict ignition. For example, in a recent conversation with a developer of the Fire Dynamics Simulator (FDS) the need was expressed for an accurate and low computational cost methodology or model to predict ignition for better fire spread predictions. Due to the large scales in fire simulations (~10\si{\meter} grid size) a detailed model of ignition (<1\si{\milli\meter} grid size) is not feasible. Similarly, risk management personnel or homeowners are not likely to have the capability to create detailed models and predictions of risk around structures and homes they are tasked with protecting. The goal and perceived impact of this work is to create or enable the creation of models that can be effectively implemented by the people interacting with fires and their risks in the field. 