%!TEX root = thesis.tex

\chapter{Literature Review}
\label{part:literature}

    Ignition of a fuel bed by firebrands consists of a multitude of processes that occur simultaneously. First, when a firebrand lands on a fuel bed heat is transferred from the firebrand to the particles of the fuel bed. As the fuel bed particles are heated they break down through a series of chemical reactions called pyrolysis. The products of the pyrolysis process, henceforth referred to as pyrolyzates, depart the fuel bed and enter the atmosphere. If the temperature and concentrations of the pyrolyzates are sufficient for reactions to occur as they mix with air ignition of the gas phase pyrolyzates occur. The flames in the gas phase then propagate to the fuel bed where the flame anchors and creates a feedback loop of preheating the fuel bed, creating pyrolyzates, and combusting them in the diffusion flame above the fuel bed. The properties of the fuel bed and the firebrand influence the characteristics of each processes. Works considering the process of ignition in biomass fuels have identified some influential properties of fuel beds that have observed to be influential to the ignition process. 
    The heating rate of the fuel bed, including mode of heat transfer significantly alters the ignition propensity of the fuel bed as the energy imparted to the fuel bed must be sufficient for pyrolysis to occur. The thermophysical properties of the fuel bed, specifally the relationship between the particle size and the thermal conductivity and their subsequent impact on the temperature evolution of the fuel bed as it is heated by the ember influences the ignition probability of the fuel bed. The influence of environmental conditions, namely wind, impacts the mixing conditions above the fuel bed affecting the ignition delay of the pyrolyzates.  Finally, the chemical composition of the fuel bed changes both the thermal and chemical properties of the fuel bed. The following sections expound on the current state of knowledge in literature for each of factors stated and provides potential avenues for research for each and what impact these research avenues would have.
    
    \section{Effect of heating rates}
        The heating rate of the fuel bed when an ember lands on it is a significant determining factor for ignition 
    \section{Effect of thermophysical properties}
    \section{Effect of environmental factors}
    \section{Effect of chemical composition}
    \begin{itemize}
        \item Get the reader to a place where they will understand the content contained in the remainder of the document
        \item Show that I have done the background work to understand and communicate the concepts contained here
    \end{itemize}
    What things need to be covered?
    \begin{itemize}
        \item Solid state combustion
        \item Heat transfer in porous media
        \item How ignition occurs 
        \item A brief history of the concepts
        \item What is currently known
        \item Methods that have been and may be used to help in this work
        \item Gaps and foundation at greater depth than the introduction
    \end{itemize}