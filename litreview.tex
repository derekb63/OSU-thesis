%!TEX root = thesis.tex

\chapter{Literature Review}
\label{part:literature}

    \subsection{Overview}
        The increasing occurrence of large wildfires and extreme fire behavior in recent years has sparked interest in obtaining a better understanding of wildfire behavior. Increasing loss of resources and structures has further motivated this interest. One of the major contributors to the increased losses is embers landing on and igniting fuel beds external from the main fire, either in a natural fuel bed in a forest or near a structure. These external ignition events are called spotfires. Of the many different aspects of fire behavior, ignition of fuels away from a fire front is one of the least predictable and least understood. When considering spotfires there are three main processes to be considered. These processes are ember generation, ember transport, and ignition of the fuel bed~\cite{Koo2010a}. Of these processes, ember transport is the best understood as it is largely a fluid dynamic phenomena. Ember generation and ignition of fuel beds are less studied phenomena, and a better understanding of these would greatly benefit the understanding of wildfire behavior~\cite{Manzello2020}. A more complete understanding of ember generation and ignition would enhance predictive capabilities of fire spread, lead to more effective prevention practices in forests and near structures, and make suppression efforts more effective. Furthering the understanding of fuel bed ignition to make these practices more effective and efficient is the overall goal of this research.
        
        Ignition of fuel beds can occur at all stages of fires, from the initial small flame that may result in a wildfire that covers 5000 square miles, to an ember that lands on a home, causes ignition, and results in the loss of a families livelihood or worse, loss of life. Despite the dangers and potential hazards that are caused by ignition, relatively little quantitative information regarding processes controlling fuel bed ignition is available. Fundamentally, the ignition of a solid fuel is quite simple: heat must be transferred to the fuel such that sufficient pyrolyzates and oxygen mix at a temperature high enough for self sustaining reactions to occur~\cite{Babrauskas2003}.
        
        In practice, the actual quantification of these parameters, such as heat transfer to the fuel bed and pyrolyzate/air mixing is quite challenging. Many studies have considered ignition of fuel beds by firebrands, but a unified theory that can encompass all or even many ignition probabilities has not been developed~\cite{Finney2013}. For example, if a smoldering pine cone lands on either a bark mulch bed or dry grass, no theory currently exists that can predict which will catch on fire a priori. Additionally, a better understanding of ignition phenomena has been recommended as one of the main focuses for wild land fire research~\cite{Manzello2018}.
        
       Taking the fundamental principles of ignition into account and looking at previous studies, a few primary parameters controlling ignition have arisen.  Fuel species, moisture content, particle morphology, and energy deposition  have been shown to affect the production of pyrolyzates. Once pyrolyzates are generated, the controlling parameters are the relative timescales between pyrolyzate/air mixing and heat loss. Air velocity over the fuel bed (i.e., wind) has been observed to have a significant impact on the mixing and heat loss of the pyrolyzates~\cite{Ellis2015}. While the effects of these parameters on ignition are interdependent the effect they have on ignition are separated and summarized as independently as possible in the following sections for clarity and completeness. 
        
        \subsection{Energy Effects}
        Fuel bed response to heating and the impacts that heat transfer has on ignition probability is important knowledge for understanding ignition of fuel beds. Unfortunately, the current state of knowledge can only answer this question qualitatively. When looking at a fuel bed's response to incident heating, there are two major parameters that can be changed: the total amount of energy deposited to a fuel bed and the rate at which energy is deposited. Intuitively, increasing both of these parameters should increase the probability of ignition. Results from literature support this reasoning where increasing the energy deposition and/or the heat transfer rate increased the probability of ignition. Table~\ref{tab:binary} provides a summary of experimental efforts that have considered the ignition probability of fuel beds. For each study in Table~\ref{tab:binary} the fuel bed material, heat source, and parameters varied are outlined. The relative increase or decrease in ignition probability for each parameter change is also identified.
  

       \begin{table}[htbp]
             \rowcolors{2}{gray!25}{white}
             \caption{Summary of studies that have considered ignition probability of fuel beds. $\downarrow$ signifies an decrease in value and $\uparrow$ an increase}
                \centering
                \begin{tabular}{N p{1.5cm} p{3cm} p{3cm} p{4.5cm} p{0.7cm}}
                \rowcolor{gray!50}
                    \multicolumn{1}{|c}{ID} &  Fuel Bed & Heat Source & Parameter(s) & $\uparrow$ Ignition Probability & Ref  \\
                     \hline
                    \label{tabrow:wang2017}&             pine needles & metal  & FMC, $T_{ember}$, $U_{wind}$ \nomenclature{$U_{wind}$}{Air velocity over the fuel bed} & $\downarrow$ FMC, $\uparrow T_{ember}$ \nomenclature{$T_{ember}$}{Temperature of the ember}, $\uparrow U_{wind}$  & \cite{Wang2017}\\
                    
                    \label{tabrow:urban2017}&            powdered grass & metal &   $T_{ember}$, $d_{ember}$ \nomenclature{$d_{ember}$}{Ember diameter} & $\downarrow d_{ember}$ requires $\uparrow T_{ember}$ & \cite{Urban2017}\\
                    
                    \label{tabrow:Fernandez-Pello2015}&  cellulose & metal &     $T_{ember}$, $d_{ember}$ & $\downarrow d_{ember}$ requires $\uparrow T_{ember}$ & \cite{Fernandez-Pello2015}\\
                    
                    \label{tabrow:ellis2015}&            forest litter & glowing or flaming wood & flaming or glowing ember, FMC,  $U_{wind}$ &  $\uparrow U_{wind}$, $\downarrow$ FMC & \cite{Ellis2015}\\
                    
                    \label{tabrow:santoni2014}&          pine needles & radiative & $A_{s}/V$, permeability & $\uparrow A_{s}/V$ \nomenclature{$A_{s}$}{Surface area of the ember} \nomenclature{$V$}{Ember Volume}, $\uparrow$ permeability & \cite{Santoni2014}\\
                    
                    \label{tabrow:reszka2012}&           nylon, PMMA & radiative & heating rate & $\uparrow$ heating rate & \cite{Reszka2012}\\
                    
                    \label{tabrow:hadden2011}&           cellulose & metal & $T_{ember}$, $d_{ember}$ & $\downarrow d_{ember}$ requires $\uparrow T_{ember}$ & \cite{Hadden2011}\\
                    
                    \label{tabrow:ellis2011}&            forest litter & flaming bark & FMC, $U_{wind}$, glowing mass &  $\downarrow$ FMC, $\uparrow$ glowing mass, $\uparrow U_{wind}$ & \cite{Ellis2011}\\
                    
                    \label{tabrow:ganteaume2009}&        forest litter & flaming wood & $\rho$, FMC, species & $\downarrow$ FMC, $\downarrow \; \rho$ , species effect & \cite{Ganteaume2009}\\
                    
                    \label{tabrow:plucinski2008}&        forest litter & cotton balls, aerial incendiary & FMC, species, $U_{wind}$ &  $\downarrow$ FMC, species effect, wind effect,  $\uparrow E_{ember}$\nomenclature{$E_{ember}$}{Energy content of ember}& \cite{Plucinski2008}\\
                    
                    \label{tabrow:manzello2006}&         litter, paper, crevices & glowing/flaming wood & $U_{wind}$, FMC, $d_{ember}$, $N_{ember}$ \nomenclature{$N_{ember}$}{Number of embers} & material effect, flaming/glowing, $\downarrow$ FMC,  $\uparrow N_{ember}$ & \cite{Manzello2006}\\
                    
                    \label{tabrow:manzello2006a}&        various mulches & glowing/flaming wood &  $U_{wind}$, FMC, $d_{ember}$ & material effect, flaming/glowing, $\downarrow$ FMC, $\uparrow U_{wind}$,  $\uparrow N_{ember}$ & \cite{Manzello2006a}\\
                    
                    \label{tabrow:yang2003}&             wood plates & radiative & heat flux & $\uparrow$ heat flux & \cite{Yang2003}\\
                    
                    \label{tabrow:delichatsios}&         wood plates & radiative & heat flux & $\uparrow$ heat flux & \cite{Delichatsios2003}\\
                    
                    \label{tabrow:dimitrakopoulos2001}&  various leaf species & radiative & FMC, species & species dependence, $\downarrow$ FMC & \cite{Dimitrakopoulos2001}
                \end{tabular}
                \label{tab:binary}
                \end{table}


        Experiments conducted by Fernandez-Pello et al.~\cite{Fernandez-Pello2015}, in which hot metal particles were dropped onto cellulose, indicated the determining factor for ignition was proportional to the energy of the firebrand. More specifically, ignition was observed if the amount of thermal energy available from the particle delivered enough energy such that sufficient pyrolyzates were generated. This relationship manifests as a parabolic relationship between the particle diameter and energy of the particle at the ignition boundary as shown in Figure~\ref{fig:energy_diameter}. 
            \begin{figure}[hpbt]
                \centering
                \includegraphics[width=0.5\textwidth]{Figures/particle_ign_boundary.png}
                \caption{Observed ignition boundary, quantified by particle energy, as a function for cellulose powder for various materials and particle sizes from Fernandez-Pello et al.~\cite{Fernandez-Pello2015}.}
                \label{fig:energy_diameter}
            \end{figure}
        Similar results for hot particles were obtained for pine needles by Wang et al.~\cite{Wang2017}, for powdered grass by Urban et al.~\cite{Urban2017}, and again with cellulose by Hadden et al.~\cite{Hadden2011}. While similar trends have been observed across a range of materials using similar energy delivery methods, efforts to create a model to predict ignition boundaries have not been successful. The difficulties preventing an ignition model have stemmed from the lack of knowledge about the ember fuel bed interface, thermal properties of the fuel bed, and difficulties in determining particle heat losses to ambient. 
        
        While observations informing trends that increased energy and heat transfer increase  probability are important, the next step is to quantify the transition or critical values for parameters observed to control the trends. A universally predictable critical value has not yet been determined but, predictive capabilities for single conditions or experimental apparatus have been determined. More detail on what has been learned about the impact of energy deposition and effects of energy deposition on pyrolysis is contained in the following sections.
        
        \subsection{Fuel Moisture Content}
        The presence of water in fuel bed particles, quantified as fuel moisture content, impacts the ignition process in multiple ways. Since the vaporization of water occurs at a lower temperature than pyrolysis, the water in a fuel particle must be evaporated before pyrolysis can occur. This creates a large energy sink for the energy imparted to the fuel bed from an ember or firebrand, thus increasing the amount of energy needed to ignite a fuel bed~\cite{Hurley2016}. Studies have universally shown this to be the case as is described in Table~\ref{tab:binary} rows~\ref{tabrow:wang2017}, \ref{tabrow:ellis2015}, \ref{tabrow:ellis2011}, \ref{tabrow:ganteaume2009}, \ref{tabrow:plucinski2008}, \ref{tabrow:manzello2006}, \ref{tabrow:manzello2006a}, and~\ref{tabrow:dimitrakopoulos2001}. This trend indicates that fuel moisture content is a parameter to be added as a lumped constant to the minimum energy needed to ignite a fuel bed. However, water in the fuel has additional impacts on the pyrolysis and ignition processes. Computational efforts have shown that increasing the moisture content of the fuel decreases both the peak mass loss rate and peak heat release rate of a fuel sample\cite{Shotorban2018}. The decrease in the heat release rate and mass loss rate may have significant effects on the chemical composition of the pyrolysis gases, as was observed by Furgeson et al.~\cite{Ferguson2013} where an increased moisture content shifted the \ce{O2} and \ce{H} concentrations in flames burning pyrolysis gases. It should be noted that this effect was observed for the piloted combustion of pyrolysis gases, but increasing amounts of \ce{H2O} would likely have a similar impact on the auto-ignition of pyrolysis gases. The implications of these effects are discussed further in Section~\ref{subsec:pyrolysis}, as they relate more directly to pyrolysis of the fuel and gas dynamics above the fuel bed. 
        
        In addition to acting as an energy sink for evaporation, increasing the fuel moisture content alters the thermal conductivity of a fuel bed. Tests for the thermal conductivity of wheat showed that a 28\% increase in wheat moisture content increased the thermal conductivity by ~40\%\cite{Tavman1998}. The increased heat diffusion rates would lead to lower temperatures and temperature gradients in the fuel bed, changing the mass flux rate of pyrolyzates departing the fuel bed and the composition of said pyrolyzates. The relative magnitude of increased moisture content in other fuel bed materials is unclear, but a ~40\% shift in thermal conductivity would have significant impacts on the heat diffusion rates from a firebrand delivering energy to the fuel bed.
        Perhaps the most important knowledge gained from the extensive study of fuel moisture content and its impacts is that due to the numerous scales across which ignition of fuel beds occur it is essential to evaluate the impacts of each parameter across all scales since the phenomena may be more closely interdependent than initially observed. 
            
        \subsection{Species and Morphology}
        One of the factors that sets the ignition of fuel beds apart from ignition of other materials is the vast array of species and particle morphology encountered in nature. For example, in a forest with a relatively uniform litter layer, the particle size distribution may range from small pieces of grass to twigs that are an order of magnitude larger. Even a layer of pine needles on a roof will have a distribution of sizes, shape, orientation, and bulk density. On the scale of the fuel bed, the differences in shapes and sizes of the particles ensure that every fuel bed is unique, making repeatability while studying fuel bed ignition difficult at best. 
        
        Implications for fuel bed uniqueness is twofold. First, the differences in particle shape, e.g., pine needles vs oak leaves, cause the heat transfer properties of fuel beds to be highly variable. Second, each species has different composition and fuel beds may consist of many species. Having fuel bed particles of different compositions and morphology in a fuel bed obfuscates the relative effects of individual particle thermal conductivity and inter-particle heat transfer due to contact area and particle shape. Furthermore, a fuel bed with multiple species will undergo a wide array of chemical processes in a single fuel bed when pyrolyzates are generated. Samples of the same fuel have also been shown to have different pyrolyzate composition under differing heating conditions~\cite{Gauthier2013}. For example, decreasing the heating source from 1050$^{\circ}$C to 450$^{\circ}$C reduced the elemental composition of hydrogen in the pyrolysis products by 23\% and increased the elemental composition of oxygen by 28\% on a \% mole basis. The inherent variability in natural fuel bed has limited observations to qualitative trends.
        
        
        Fortunately, despite the large variability across species and materials, there are three major components that are common across most biomass fuels: cellulose, hemicellulose, and lignin. Recent efforts, supported by the large amount of data produced by the previously mentioned studies, have attempted to find a common characterization for biomass materials by comparing composition differences. It was found that characterization of cellulose, hemicellulose, and lignin composition was not sufficient and extractive compounds must be considered. When extractive compounds were considered, chemical kinetic properties were able to be predicted within accuracy of existing models~\cite{Debiagi2015} with more detailed characterizations. This development helps lessen some of the difficulties encountered in previous studies by enabling the comparison of different species with less intensive methods of characterization. The ability for this approach to increase comparability between tests is a significant step forward in the understanding of ignition. However, there remains a multitude of effects on ignition introduced by species and morphology that need to be considered. These effects are outlined and considered in the following section. 
        
        \subsection{Pyrolysis Production and Air Mixing}\label{subsec:pyrolysis}
        Pyrolysis of fuel bed material is the keystone process for the ignition of fuel beds. Once the heat transfer to the fuel bed raises the temperature of the fuel enough for pyrolysis to occur, a number of conditions must be met for the ignition to occur.  The thermal degradation reactions that occur at the onset of pyrolysis are heavily influenced by the parameters discussed in the previous sections. The rate of energy deposition, fuel moisture content, species, and fuel morphology all modify the rate and composition of pyrolysys products. 
        The heating rate and the overall energy imparted to the fuel bed have the most prominent effect on the pyrolysis reactions in a fuel bed. At low temperatures and low heating rates the main components released by the fuel bed consist of chars, tars, and other large hydrocarbon chain molecules due to incomplete breakdown of the materials. As temperatures increase the products shift to more complete reactions that increase the amount of \ce{CO}, \ce{CO2}, \ce{H2}, and small hydrocarbons produced~\cite{Gauthier2013}. An example of this effect is shown in Figure~\ref{fig:temp_effects_pyrolysis} where small wood particles were placed in a heating apparatus at various temperatures. 
            \begin{figure}
                \centering
                \includegraphics[width=0.75\textwidth]{Figures/gauthier2013_temp_effects.png}
                \caption{Pyrolysis gas concentrations for beech wood samples heated at different rates from ~\cite{Gauthier2013} showing the significant decrease in gas production rates for most gases as the sample temperature decreases.}
                \label{fig:temp_effects_pyrolysis}
            \end{figure}
        As is plotted in Figure~\ref{fig:temp_effects_pyrolysis}, the heating rate affects both the rate of gas productions and the total amount of gas production. These trends have significant implications for ignition of pyrolysis gases above the fuel bed, since auto-ignition is dependent on the temperature, concentration, and species of the gases in the buoyant plume above the fuel bed.
        
        Similar trends have been observed for larger solid plates of plastics and wood in a wind tunnel with piloted flame. It was observed that under higher heat fluxes, the ignition time was shorter but the amount of pyrolyzates generated per time increases. The rate of pyrolyzate generation above which ignition was observed is considered to be the critical mass flux. This phenomena has been observed consistently across plastic and wood plates, as well as pine needles~\cite{McAllister2013, Hernandez2018}. The increase in the critical mass flux for ignition is attributed to higher temperature gradients in the material, causing most of the chemical reactions to occur near the surface of the material at a higher oxygen content and temperature~\cite{Rich2007}. Reactions under these condition produce more \ce{CO} and \ce{CO2}, as shown in Figure~\ref{fig:temp_effects_pyrolysis}, which have a increased lower flammability limit than larger hydrocarbons produced at lower heat fluxes. What is not clear from these findings is how fuel beds of different materials compare to solid fuels under the same conditions or the effects of different heat transfer modes. As particle sizes increase, the properties of the fuel bed become further from a solid as the porosity, surface area, and oxygen availability increase. The converse of these increases is a likely drop in bulk fuel bed thermal conductivity, creating higher temperature gradients. These temperature gradients have been shown to have change pyrolyzate composition for very small sample sizes ($\leq$10mg) in highly controlled Thermogravametric Analysis (TGA) tests~\cite{Richter2018}. While it is apparent that temperature gradients in the fuel bed affect pyrolyzate production, it is unclear how this will effect ignition of the fuel bed on scales found in fires. For these larger fuel beds, distributions of temperature and species concentrations caused by non-uniform and multiple heat sources is an important factor to understand when considering ignition. 
        
        Entrainment and mixing of air with pyrolyzates above the fuel bed has a pronounced effect on the ignition of fuel beds. As shown in Table~\ref{tab:binary} rows~\ref{tabrow:wang2017}, \ref{tabrow:ellis2015}, \ref{tabrow:ellis2011}, \ref{tabrow:plucinski2008} studies of fuel bed ignition that considered variable wind speed reported that an increase in wind speed increased the probability of ignition with other factors held constant. Increasing the flow velocity above the fuel bed reduces both the relative importance of buoyant forces and the timescale for mixing. With increased velocity and turbulence, air and pyrolyzates mix faster but heat loss to increased air advection also increases. The relative magnitude of time scales between heat and species diffusion in reacting flows are commonly compared using the Damkohler number as shown in Equation~\ref{eqn:dam}.
        
            \begin{equation}\label{eqn:dam}
                \text{Da} = \frac{\text{flow time scale}}{\text{chemical time scale}}
            \end{equation}
        In the Damkohler number, the flow velocity and turbulence that affect the mixing of air and pyrolyzates are considered in the flow time scale term.  The chemical time scale term considers the pyrolyzate species concentration and temperature effects. The interaction of these two terms is well defined and commonly used in combustion to determine flame regimes and ignition. Use of the Damkohler number as a relation to define ignition has seen limited use in ignition of fuel beds since the knowledge of species concentrations, temperatures, and air flow effects are largely unknown. Gaining insight into these effects would enable the use of tools like the Damkohler number to successfully predict ignition across a variety of fuel beds. 
        Namely, quantification of the parameters near the fuel bed (e.g., surface wind speed), fuel properties (e.g., chemical composition, estimated ember contact area), and the energy content of a firebrand may enable a common metric to predict if ignition will occur in a fire without ignition testing for a certain specific fuel, set of conditions, and location.
        \subsection{Summary}
        Substantial progress has been made to further the understanding of fuel bed ignition and the effects of various related parameters. Despite these efforts, there remains no universal or even quantitative approach that can be used to effectively predict ignition of fuel beds without prior testing for a specific source or fuel. Current knowledge is capable of predicting ignition of solids and predicting different pyrolysis properties of fuels, but the knowledge has not been extended to fuel beds that may occur in forest fires and near structures. One of the largest gaps in knowledge that remains is to determine how fuel beds that consist of fine fuel particles behave in comparison to solid theory. In addition, the effects of non-uniform heating and variable energy deposition are largely unknown. In order to close these knowledge gaps three avenues of research are proposed. First, the heat transfer properties of the fuel bed must be quantified to obtain better knowledge of how the fuel bed responds to heating. Second, the generation rates of pyrolyzates and the effect that heating rates, heat source temperatures, and heating modes must be quantified. Third, the mixing timescales between the pyrolyzates and air must be evaluated to determine what rate the pyrolyzates must be generated to achieve ignition across a variety of conditions. This approach will help bridge the gap between the fundamental theories for ignition and real fuel beds where wildfires occur. The approach and methodologies for each of these fundamental processes is outlined in the following sections. 