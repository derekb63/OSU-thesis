%!TEX root = thesis.tex

\renewcommand{\TheTitle}{Effect of Fuel Bed Composition on Flaming Ignition Probability}
\renewcommand{\TheAuthors}{Derek Bean, David L. Blunck \\ \\ My contributions to this work included the design of the experiments, fabrication of the experimental apparatus, collecting data, data analysis, and preparation of the manuscript.}

\renewcommand{\TheAddress}{
\textbf{Status: In Preparation}\\
Target Journals: Fire Safety Science, Fire Safety Journal, or Fire Technology
}

\PaperHeader{\chapter{\TheTitle}\label{chap:manuscript3}}{\TheAuthors}{\TheAddress}

\newpage


\section{Abstract}
\label{sec:abtract3}
    The increasing severity of wildfires and the expansion of the wildland urban interface have placed a greater number of homes at risk of being destroyed by wildfires. Ignition of fuel beds on or near homes by firebrands is a significant source of fire spread and structure loss. Fire prevention and suppression efforts lack an expedient method for estimating the likelihood of ignition for the wide variety of materials around homes. In this study, the effects of fuel bed chemical composition on the temperature threshold for flaming ignition was considered. Fuel beds of six different species were heated with a cartridge heater as a surrogate firebrand. Tests were conducted in both quiescent and windy conditions. Fuel beds with high lignin content (i.e., Douglas-fir bark and pine park) were not observed to ignite at the maximum temperature at either wind speed. Increasing the wind speed decreased the ignition threshold for Douglas-fir wood and pine wood but increased the threshold for oak wood. Decreases in the ignition threshold are attributed to recirculation zones near the fuel bed and firebrand interface. However, the increase in ignition threshold for oak wood suggests that chemical sensitivities may inhibit ignition enhancement due to wind. This work suggests that areas of high risk of firebrand exposure may reduce the risk of ignition by using materials high in lignin content. 
        
\section{Introduction}\addvspace{10pt}
\label{sec:introduction3}
    
    After a firebrand lands on a fuel bed, the parameters that determine if ignition occurs can be broken down into three categories. The first category contains the properties of the firebrand that lands. The second category considers the heat transfer between the firebrand, fuel bed, and the surroundings. The last category is how the fuel bed responds to the heat transfer from the firebrand. The response of the fuel bed to the heat transfer from the firebrand is influenced by the morphology and chemical composition of the particles in the fuel bed. Fuel bed materials present in wildfires vary significantly in both morphology and chemical composition, and knowledge of the effects of each on ignition is qualitative, at best. A previous study evaluated the impact of particle morphology~\cite{Bean2021}, and this work seeks to identify how differences in the chemical composition of a fuel bed under attack by firebrands influence the probability of flaming ignition. 
    
    Fuel beds at risk for ignition by firebrands vary significantly in chemical composition which may result in differences in thermal properties and the chemical species produces during pyrolysis. Works considering ignition of fuel beds with varying concentrations of constituents have found significant variation in ignition characteristics. Cellulose fuel beds have been observed to have a lower ignition threshold when compared to grass and pine needles when put in contact by hot metal particles~\cite{Urban2018}. Interestingly, cellulose undergoes pyrolysis at either higher temperatures or later in the heating process than both lignin and hemicellulose~\cite{Yang2007a, Shotorban2018}. This suggests that the lower ignition temperature of cellulose is not due to pyrolysis occurring at lower temperatures, but to the flammability of pyrolyzates produced. 
    
    One method to evaluate the effects of chemical composition on the flammability of pyrolyzates produced, and thus ignition propensity, is through modeling of the pyrolysis process. Efforts to characterize the pyrolysis of biomass have continually improved the understanding of the chemical processes that occur in both the solid and gas phase of pyrolysis products. Recently, more inclusive pyrolysis models have been developed that improve on the initial single step Arrhenius reactions~\cite{DIBLASI199371} and more closely approximate the reactions occurring during combustion of biomass~\cite{Ranzi2008, Debiagi2015, Dhahak2019} based on the initial chemical composition of the fuels. With the development of models that are applicable to a wide range of chemical compositions and environmental factors the possibility of modelling pyrolysis and ignition of fuel beds is possible. Coupling these models with databases, such as the Bioenergy Feedstock Library~\cite{feedstock} (which contains the chemical composition information for a wide variety of materials that may be vulnerable to firebrand attack during a wildfire), predictions of ignition potential may be possible without data from ignition experiments. While currently available tools, like combustion models and composition databases, may be useful individually for understanding fuel bed ignition and pyrolysis, a framework linking the individual resources and knowledge to provide a general model of fuel bed ignition is lacking. Unfortunately, the current level of knowledge on the ignition threshold of materials of variable composition does not facilitate such a framework.
    
  Studies considering ignition have primarily focused on the influence of heat transfer and fuel bed conditions when identifying ignition parameters. The observed sensitivities to fuel bed properties are attributed to heat transfer and chemistry-related processes. The probability of ignition generally increases as the energy content of a firebrand increases~\cite{Hadden2011}. However, further research has indicated that energy content alone is not a sufficient indicator of ignition as a minimum firebrand temperature (a driver of heat flux rates) is also necessary~\cite{Zak2014}. Additionally, heat loss from firebrands to the surroundings has a significant effect on ignition~\cite{Fernandez-Pello2015}. While the previous efforts have focused primarily on identifying the impact of firebrand properties on ignition, efforts focused on the properties of the fuel bed have shown sensitivities to fuel bed material type and particle size~\cite{Urban2018}. 

    Additionally, for fuel beds of constant particle size, an increase in the packing density of the fuel bed results in a decrease in the critical heat flux needed for ignition~\cite{Mindykowski2011, Hernandez2017, Rivera2020}. The observed sensitivities to fuel bed properties are likely caused by differences in both heat transfer and chemistry-related processes. Changes in the particle size may impact the contact area (e.g., a ball resting on fine sawdust compared to a pile of pine needles) between the firebrand and fuel bed, potentially reducing the amount of heat transferred through conduction as the contact area decreases. Additionally, when considering a fixed particle size fuel bed, changes in chemical composition may change the heat transfer and heating rate of the materials due to changes in the thermal conductivity of the material affecting the ignition properties of the material. For these reasons, knowledge of the effects of chemical composition on ignition, as it relates to thermal and chemical processes, must be addressed to accurately represent ignition properties for the wide range of materials found in fires resulting in home loss.
  
    The objective of this work is to identify how changes in chemical composition of fuel bed materials effect the probability of flaming ignition. It is anticipated that knowledge gained from this study may be used improve the understanding of fuel bed ignition for different fuels and help build a framework for estimating ignition probabilities of different fuels without the need for extensive laboratory testing.

\section{Methods}
\label{sec:methods3}
    The 50\% probability of flaming ignition with respect to the surface temperature of a resistance heater was determined for five different materials in quiescent conditions and at a wind speed of 5.8\si{\meter\per\second}. The fuel bed materials used were Douglas-fir wood, Douglas-fir bark, pine wood, pine bark, wheat straw, and oak wood. These materials were chosen to represent a range of chemical compositions, and are materials that may be subject to firebrand attack in a wildland urban interface fire. Table \ref{tab:composition} shows the estimated chemical composition of the materials based on the Bioenergy Feedstock Library~\cite{feedstock}. To create fuel beds of consistent particle size,  materials were granulated and then sorted such that the particles fit through a screen with openings of 2.1\si{\milli\meter} but not through openings of 0.85\si{\milli\meter}. Once sorted, the materials were oven dried at 103\si{\celsius}.
    \begin{table}[hpbt]
        \caption{Proportion of cellulose, hemicellulose, and lignin of the materials tested estimated from the Bioenergy Feedstock Library~\cite{feedstock}}
        \centering
        \begin{tabular}{crrrr}
            % \rowcolor{gray!50}
            Material & Cellulose (\%) & Hemicellulose (\%) & Lignin (\%) & Other (\%) \\
            \hline
            Oak Wood         & 45 & 22 & 24 & 9\\
            Douglas-fir Wood & 44 & 22 & 28 & 6\\
            Wheat Straw      & 44 & 27 & 19 & 10\\
            Pine Wood        & 40 & 30 & 20 & 10\\
            Pine Bark        & 39 & 17 & 37 & 7\\
            Douglas-Fir Bark & 26 & 11 & 59 & 4 
        \end{tabular}
        \label{tab:composition}
    \end{table}
        \begin{table}[hpbt]
        \caption{Average bulk density of materials for each material and wind speed.}
        \centering
        \begin{tabular}{crr}
            % \rowcolor{gray!50}
            Material & U\textsubscript{bulk} = 0.1\si{\meter\per\second} (\si{\kilogram\per\cubic\meter})& U\textsubscript{bulk} = 5.8\si{\meter\per\second} (\si{\kilogram\per\cubic\meter})\\
            \hline
            Oak Wood         & 344 & 428 \\
            Douglas-fir Wood & 99  & 74 \\
            Wheat Straw      & 85  & 85 \\
            Pine Wood        & 114 & 115 \\
            Pine Bark        & 215 & 172 \\
            Douglas-Fir Bark & 131 & 146  
        \end{tabular}
        \label{tab:density}
    \end{table}
    Two experimental apparatus were used to conduct ignition experiments. The experimental apparatus used for the tests in quiescent conditions is shown in Figure~\ref{fig:speciesApparatus}. The quiescent apparatus used a lever arm to hold the cartridge heater in a fixed position the duration of the test. The heater was inserted into the fuel bed approximately half of the heater diameter (3\si{\milli\meter}). Tests conducted with a wind speed of \si{\meter\per\second} were performed using the apparatus shown in Figure~\ref{fig:speciesWindTunnel}. This apparatus was operated inside of a wind tunnel where the heater was lowered using an automated lowering device that maintained a constant pressure equivalent to that of a 10\si{\gram} firebrand landing on the fuel bed. The pressure was monitored and adjusted using a PID control system with force measured by a load cell. For all tests the cartridge heater (firebrand surrogate) used was a 50\si{\milli\meter} long 6.35\si{\milli\meter} diameter cartridge heater.
        \begin{figure}[htpb]
        \centering
        \resizebox{\figureWidthSet}{!}{%
        \begin{tikzpicture}
            \filldraw [draw=black!60, pattern=north west lines, pattern color= brown] (0, 0) rectangle (140mm, 70mm) node[pos=0.5, fill=white] {\Huge Fuel Bed}; 
            \filldraw[] (0mm, -2mm) rectangle (140mm,0mm) node[rotate=-90, pos=0.5] {};
            \filldraw[] (-2mm, -2mm) rectangle (0mm,70mm) node[rotate=-90, pos=0.5] {};
            \filldraw[] (140mm, 70mm) rectangle (142mm,-2mm) node[rotate=-90, pos=0.5] {};
            \filldraw [fill=red, draw=black] (45mm, 67mm) rectangle (95mm, 73mm) {}; 
            \filldraw [draw=black, fill=black!50] (55mm, 65mm) rectangle (58mm, 85mm) {};
            \filldraw [draw=black, fill=black!50] (66mm, 85mm) rectangle (48mm, 185mm) {};
            \filldraw [draw=black, fill=black!50] (-85mm, 220mm) rectangle (-67mm, -2mm) {};
            \filldraw [draw=black, fill=black!50, rotate around={-15:(57mm, 100mm)}] (66mm, 95mm) rectangle (-100mm, 105mm) {};
            \filldraw [draw=black, fill=black!50, rotate around={-15:(57mm, 170mm)}] (66mm, 165mm) rectangle (-100mm, 175mm) {};
            \filldraw [draw=black, fill=black!50] (-90mm, -2mm) rectangle (150mm, -4mm) {};
            \filldraw (57mm, 170mm) circle (2mm);
            \filldraw (57mm, 100mm) circle (2mm);
            \filldraw (-76mm, 205mm) circle (2mm);
            \filldraw (-76mm, 135mm) circle (2mm);
            \filldraw [rotate around={15:(140mm, 85mm)}] (140mm, 85mm) rectangle (145mm, 90mm) node[above right] {\Huge Photodiode};
            \draw [<-, line width=1mm] (75mm, 75mm) -- (125mm, 120mm) node[right] {\Huge Heater};
            \draw [dashed] (140mm, 87.5mm) -- (100mm, 72mm);
            \draw [dashed] (140mm, 87.5mm) -- (80mm, 74mm);
        \end{tikzpicture}
        }
        \caption{Experimental apparatus for the ignition tests. The lever arm used to lower the apparatus into the fuel bed, the fuel bed size relative to the heater, and the location of the photodiode are illustrated.}
        \label{fig:speciesApparatus}
    \end{figure}
    The temperature of the heater was recorded with a type-K thermocouple for the duration of each test. The temperature of the heater for ranged from 250\si{\celsius} to 750\si{\celsius} and was controlled using PID control logic implemented in LabVIEW. The PID controller kept the heater temperature within 6\% of the setpoint for the duration of the tests. The temperature setpoints of the resistance heater were determined using the three-phase optimal design procedure to efficiently determine the desired ignition probability for the number of tests conducted~\cite{Wu2014}. For tests where flaming ignition did not occur data was recorded for 3000\si{\second} or until the reaction front of the smoldering material reached the edge of the test container. The fixed position of the cartridge heater and the use of the cartridge heater was implemented to maintain consistency between tests of the same material and between test series of different materials. The use of the cartridge heater and the fixed position is not necessarily representative of a burning firebrand landing on the fuel bed but the consistency of surface temperatures, ability to record temperature, and consistent contact with the fuel bed are essential for comparing results across test series and materials. For the quiescent tests flaming ignition was detected with the use of a BPX 65 photodiode sampled at 1 kHz. If the intensity detected rose above a set threshold flaming ignition was said to occur. For the wind tunnel tests flaming ignition was determined from the rising temperature of the cartridge heater in the presence of flame. Visual detection of flames was also used in cohort with the photodiode measurements. The 50\% probability of ignition was determined from a logistic regression on the outcomes of 25 ignition tests conducted for each of the materials for which flaming ignition was observed. The scikit-learn python package~\cite{scikit-learn} was used to perform the logistic regressions used to predict ignition probabilities.
    \begin{figure}[hbpt]
            \centering
            \resizebox{0.5\columnwidth}{!}{%
                \begin{tikzpicture}
                    \filldraw[pattern=north west lines, pattern color=brown, thick] (2.84, 0)  rectangle (4.34, -.75) node[pos=0.5,rectangle,fill=white] {\scriptsize Fuel Bed};
                    \fill [draw=white,  inner color=red, outer color=white ] (3.59, 0.01) circle (0.15);
                    \filldraw[draw=black,fill=white, thick] (0, 0)      rectangle (6, 3);
                    \fill[fill=black!50] (3.58, 0) rectangle (3.61, 1.52);
                    \fill[fill=black] (3.55, 1.52) rectangle (4.23, 1.57);
                    \filldraw[draw=black, fill=black!50] (4.09, 1.57) rectangle (4.22, 3);
                    \fill[draw=red, fill=red] (3.59, 0.01) circle (0.0635);
                    \draw [<-] (3.5, 0.1) -- (3, 0.5) node[left] {\scriptsize Heater};
                    \draw [<-] (3.5, 1.55) -- (3, 1.55) node[left] {\scriptsize Load Cell};
                    \draw [<-] (3.75, 2.5) -- (3, 2.5) node[left] {\scriptsize Lowering Arm};
                    \draw[->]         (0.1, 0.6) -- (0.75, 0.6);
                    \draw[->]         (0.1, 1.2) -- (0.75, 1.2);
                    \node[right] at (-0.75, 1.5) {\scriptsize Inlet};
                    \draw[->]         (0.1, 1.8) -- (0.75, 1.8);
                    \draw[->]         (0.1, 2.4) -- (0.75, 2.4);
                \end{tikzpicture}
                }
            \caption{Diagram of the experimental wind tunnel apparatus. Air flows through the wind tunnel from left to right.}
            \label{fig:speciesWindTunnel}
        \end{figure}
    Table~\ref{tab:solidProperties} shows the estimated thermal properties for solid materials, as obtained from literature. 
    \begin{table}[hpbt]
        \caption{Estimated material properties the fuel beds tested. The properties of the solid materials were obtained from literature. The thermal conductivity ($k_{bed}$) and thermal diffusivity ($\alpha_{bed}$) were calculated.}
        \centering
        \begin{tabular}{crrr}
            % \rowcolor{gray!50}
            Material &
            $k_{solid}$ (\si{\watt\per\meter\per\kelvin})&
            $\rho_{solid}$ (\si{\kilo\gram\per\cubic\meter}) &
            $c_{solid}$ (\si{\kilo\joule\per\kilogram\per\kelvin})\\
            \hline
            Douglas-fir Wood & 0.12~\cite{Laboratory2010} & 510~\cite{Bean2021}  & 1.26~\cite{Laboratory2010}\\
            Pine Wood        & 0.10~\cite{Laboratory2010} & 400~\cite{Miles2009} & 1.36~\cite{Laboratory2010}\\
            Oak Wood         & 0.15~\cite{Laboratory2010} & 600~\cite{Miles2009} & 1.23~\cite{Laboratory2010}\\
            Wheat Straw      & 0.16~\cite{Mason2016}      & 1038~\cite{Lam2007}  & 1.34~\cite{Stenseng2001}  \\
            Pine Bark        & 0.21~\cite{Gupta2003}      & 350~\cite{Miles2009} & 1.36~\cite{Gupta2003}     \\
            Douglas-Fir Bark & 0.21~\cite{Gupta2003}      & 440~\cite{Miles2009} & 1.36~\cite{Gupta2003}     
        \end{tabular}
        \label{tab:solidProperties}
    \end{table}
    The thermal conductivity of the porous fuel bed materials was calculated using correlations for porous media from literature~\cite{bergman2011fundamentals}. Equation~\ref{eqn:speciesk} shows the the correlation used where $\epsilon$ is defined as the proportion of volume occupied by air, as is shown in Equation~\ref{eqn:speciesEpsilon}.
            \begin{equation}
            k_{eff} = \frac{1}{2} \left(\frac{1}{\left( 1 - \epsilon \right)/k_{solid} + \epsilon/k_{air}} +  \epsilon k_{air} + \left(1-\epsilon \right) k_{solid} \right)
            \label{eqn:speciesk}
        \end{equation}
        \begin{equation}
            \epsilon = 1- \frac{\rho_{solid}}{\rho_{bed}}
            \label{eqn:speciesEpsilon}
        \end{equation}
    The specific heat of the porous media was determined using the respective proportions of solid and air, as defined by $\epsilon$, using the values obtained from literature.
        \begin{equation}
            c_{bed} = \epsilon \; c_{air} + \left( 1-\epsilon \right)\; c_{solid}
            \label{eqn:speciesC}
        \end{equation}
    The calculated thermal conductivity ($k_{bed}$)  and diffusivity values ($\alpha_{bed}$) are shown in Table~\ref{tab:speciesKValues}.
    \begin{table}[hpbt]
        \caption{Calculated thermal conductivity ($k_{bed}$) and thermal diffusivity ($\alpha_{bed}$) for the materials tested at both wind speeds.}
        \centering
        \begin{tabular}{crr|rr}
            \rowcolor{white}
            & 
            \multicolumn{2}{c}{U\textsubscript{bulk} = 0.1\si{\meter\per\second} } & \multicolumn{2}{c}{U\textsubscript{bulk} = 5.8\si{\meter\per\second} } \\
            \rowcolor{white}
            \multirow{-2}{*}{Material} &
            \parbox{2.0cm}{$k_{bed}$ (\si{\watt\per\meter\per\kelvin})}&
            \parbox{2.0cm}{$\alpha_{bed}\cdot$10\textsuperscript{-7} (\si{\square\meter\per\second})} &
            \parbox{2.0cm}{$k_{bed}$ (\si{\watt\per\meter\per\kelvin})} & 
            \parbox{2.0cm}{$\alpha_{bed}\cdot$10\textsuperscript{-7} (\si{\square\meter\per\second})}\\
            \hline
            Douglas-fir Wood & 0.038 & 4.72 & 0.035 & 5.49 \\
            Pine Wood        & 0.040 & 4.87 & 0.040 & 4.84\\
            Oak Wood         & 0.074 & 5.07 & 0.089 & 7.17 \\
            Wheat Straw      & 0.033 & 4.20 & 0.033 & 4.20 \\
            Pine Bark        & 0.098 & 11.7 & 0.081 & 9.18 \\
            Douglas-Fir Bark & 0.058 & 6.25 & 0.062 & 6.30 
        \end{tabular}
        \label{tab:speciesKValues}
    \end{table}
    
\section{Results and Discussion}
    \label{sec:results3}
    The flaming or non-flaming ignition outcome of each test and the logistic regression results for Douglas-fir wood, oak wood, and pine wood are shown in Figure~\ref{fig:logistic_plot} for the quiescent tests. The circular markers represent the outcome of each test, either flaming or no ignition. The logistic regression for each material is shown as a solid line with the shaded regions representing the 95\% confidence interval for each regression. Douglas-fir bark, pine bark, and wheat straw are not shown in Figure~\ref{fig:logistic_plot} because flaming ignition was not observed for five tests at the maximum heater temperature of 750\si{\celsius}. The heater temperature estimated to produce 50\% ignition probability from the logistic regressions for each material is shown in Figure~\ref{tab:composition50temp}. Anecdotally, self-sustained smoldering was observed for all five materials at temperatures lower than the flaming ignition temperature. The heater temperature corresponding to the onset of smoldering ignition was not measured in this study. 
        \begin{figure}[htpb]
            \centering
            \includegraphics[width=\figureWidthSet, trim={0 0 37cm 0}, clip]{conference_results_binary.png}
            \caption{Test results for materials where flaming ignition was observed. The circular markers denote individual tests. The solid lines represent the logistic regression and the shaded zones the 95\% confidence interval with the 'x' denoting the temperature for 50\% ignition probability.}
            \label{fig:logistic_plot}
        \end{figure}
    Two aspects of the ignition results are of note from the quiescent cases. Differences in the heater temperature estimated to produce 50\% ignition probability suggest that significant differences in ignition occur across a range of materials in the same apparatus and experimental conditions. Second, the transition between no ignition and flaming is less abrupt (i.e., the 95\% confidence interval of the regression spans a much larger temperature range) in the Douglas-fir results when compared to oak wood and pine wood. This is significant because it may cause more difficulty predicting the ignition of materials that ignite at lower temperatures. Lower confidence in predicting ignition for more easily ignitable materials would be detrimental to the usefulness of a model or predictive tool that may be implemented from this testing methodology. It is, however, unclear if the ignition to no ignition transition in other materials of similarly low ignition temperature will behave similarly to that of Douglas-fir. Characterization of additional materials is needed to determine if the ignition to no ignition transition becomes less abrupt as the ignition threshold decreases. 
    
    Figure~\ref{fig:composition_plot} shows the averaged cellulose, hemicellulose, and lignin concentrations for each material according to the samples recorded in the Bioenergy Feedstock Library, as shown in Table~\ref{tab:composition}. The marker colors for each material correspond to the calculated 50\% ignition probability for the quiescent tests except for the wheat straw and Douglas-fir bark, which are denoted as 800\si{\celsius} to indicate that the 50\% ignition probability was higher than the temperatures tested. Note that due to scaling inherent to a ternary projection, the proportion of the fuels that is not cellulose, hemicellulose, or lignin, causes a shift in the data. For example, the wheat straw and Douglas-fir are estimated to have the same cellulose content but do not lie on the same iso-line of 44\% cellulose. Nonetheless, the relative proportions of each constituent are retained, and comparisons may still be made. From the data shown in Figure~\ref{fig:composition_plot} a local minimum ignition threshold exists near the concentration of Douglas-fir wood. If the apparent local minimum persists with additional materials, it could be used to rapidly identify materials less conducive to ignition which would be a powerful tool for choosing appropriate materials on and around homes in areas at risk of ember attack. 
        \begin{figure}[htpb]
            \centering
            \includegraphics[width=0.75\columnwidth]{Figures/speciesStudyTernaryDiss.png}
            \caption{Estimated chemical composition of each material tested with the marker color representing the estimated 50\% ignition probability. Note that the materials where flaming ignition did not occur are represented as 800\si{\celsius} to show that the temperature of ignition was not achieved.}
            \label{fig:composition_plot}
        \end{figure}
    
    The 50\% ignition probability results for the 5.8\si{\meter\per\second} wind speed tests are shown in Table~\ref{tab:composition50temp} along with the results for the 0.1\si{\meter\per\second} tests. From these results, there are three observations of note. First, the increase of wind lowered the threshold for ignition probability for the Douglas-fir wood and Pine wood by approximately 30\%. The decreased threshold for ignition  is attributed to the formation in recirculation zones near the heater. This is consistent with results presented in Chapter~\ref{chap:manuscript2}. Second, in contrast to the Douglas-fir and Pine wood results, an increase in wind resulted in an increase in the ignition threshold for oak wood. Flaming ignition was observed for some tests of oak wood under wind. However, the predicted threshold for ignition was above 750\si{\celsius}. Third, the materials that did not ignite in the quiescent tests were also not observed to ignite in the presence of wind. The trends for Douglas-fir wood and pine wood match those presented in Chapter~\ref{chap:manuscript2} and trends reported in other studies of pine needle ignition~\cite{Wang2017} and eucalyptus bark~\cite{Ellis2011, Ellis2015}. However, the decreasing ignition probability of the oak wood contradicts those observed in the aforementioned studies and this study. A decrease in the ignition in the presence of wind is not unprecedented. An ignition study of various litter layer types ignited by different types of firebrands observed sensitivity to firebrand location. Ignition probability was reduced when the firebrand landed on top of the fuel bed when compared to embedded firebrands~\cite{Plucinski2008}. The decrease in ignition probability was attributed to increased heat loss due to wind. In this study, however, the location of the cartridge heater was consistent across all materials, the energy is not a function of the wind speed, and the flow field around the firebrands is consistent.      
        \begin{table}[hpbt]
            \caption{Heater temperature required for 50\% probability of ignition for the materials and conditions tested.}
            \centering
            \begin{tabular}{crrr}
                % \rowcolor{gray!50}
                Material & 0.1 \si{\meter\per\second}(\si{\celsius}) & 5.8 \si{\meter\per\second} (\si{\celsius}) & $\Delta$T (\si{\celsius})\\
                \hline
                Douglas-fir Wood & 574  & 391 & -185 \\
                Pine Wood        & 634  & 430 & -204\\
                Oak Wood         & 663  & \textgreater750 & 87\\
                Wheat Straw      & \textgreater750 & \textgreater750 & -\\
                Pine Bark        & \textgreater750 & \textgreater750 & -\\
                Douglas-Fir Bark & \textgreater750 & \textgreater750 & -
            \end{tabular}
            \label{tab:composition50temp}
        \end{table}
    Considering the consistent ignition heat source and environmental conditions, the remaining differences that may be driving factors for ignition trends of oak wood are material properties or changes in pyrolyzate ignition due to wind. The most accessible material property to quantify and compare is the bulk density of the materials. However, there is no clear correlation between density and the probability of ignition. Consider first the wheat straw, which did not ignite at either wind speed, had a bulk density between pine wood and Douglas-fir wood, which ignited in both wind conditions. Based on bulk density, wheat straw is anticipated to ignite at a temperature between these two materials. Similarly, oak wood has a bulk density higher than the Douglas-fir bark and pine bark but was observed to ignite in the low wind speed cases. Clearly, the bulk density of the material is not sufficient to differentiate between cases where ignition does or does not occur. 
    
    The thermal diffusivity of the fuel bed materials may provide insight into ignition trends that bulk density does not capture. From the thermal diffusivity values presented in Table~\ref{tab:speciesKValues} the fuel bed materials with the highest thermal diffusivity values (pine bark, Douglas-fir bark, and oak wood in a wind speed of 5.8\si{\meter\per\second}) were not observed to ignite. While there appears to be a correlation between thermal diffusivity and ignition, the wheat straw had a lower thermal diffusivity than both pine wood and Douglas-fir wood but was not observed to ignite in any tests. It is possible that, similar to the chemical composition results, an ideal thermal diffusivity for ignition exists near the thermal diffusivity of Douglas-fir wood and pine wood, however a more likely cause is error from the multiple correlations used to estimate values or differences in measurements across the multiple studies from which properties were collected. Thus, neither the bulk density nor the thermal diffusivity estimates made here are sufficient to explain differences in ignition, leaving differences in chemical composition as the remaining predictor.  
    
    Douglas-fir bark and pine bark, with the highest lignin content, were not observed to ignite for the conditions tested. This suggests that materials with high lignin contents may be the least likely to experience flaming ignition when exposed to firebrands and thus are the safest with respect to risk around homes. However, this observation does not consider the potential for smoldering ignition and a subsequent smoldering to flaming transition that may occur. Thermogravimetric analysis experiments have shown that while lignin begins to decompose at the lowest temperature of the three primary components, it decomposes at the slowest rate and over a much wider range of temperatures~\cite{Yang2007a}. It was also observed that the peak gas production of \ce{CO}, \ce{CO2}, and \ce{CH4} occurs at higher temperatures than hemicellulose and cellulose. In contrast to Douglas-fir bark, wheat straw had the lowest estimated lignin concentration but was also not observed to ignite. When comparing wheat straw to Douglas-fir wood, the higher lignin concentration suggests that Douglas-fir wood is likely to have a higher ignition temperature than wheat straw. This discrepancy in observed ignition behavior and what is expected based lignin content is attributed to two material attributes. First, the pyrolysis process is complex, and there is not a clear linear relationship between the composition of the fuel and flaming ignition. It appears that materials of different compositions are capable of producing gaseous products of near equal ignitability. However, the proportion of each constituent for which ignition occurs most readily is unclear. Second, the thermal conductivity of the materials likely varies significantly, which impacts the temperature gradients and mass of the material above the pyrolysis temperature, further obfuscating the effect of composition on ignition. 
    
\section{Conclusions}
\label{sec:conclusions3}    
    Flaming ignition tests were conducted for fuel beds of six different materials at two different wind speeds. Ignition of fuel beds was induced by a cartridge heater that served as a firebrand surrogate. The temperature of the heater was maintained at a fixed temperature throughout each test allowing sensitivities of ignition to fuel bed composition to be evaluated. The ignition probability and chemical composition were used to evaluate the influence of chemical composition of the fuel bed and observe differences in ignition probabilities in the presence of wind and in quiescent conditions. The specific conclusions of this work are as follows:
        \begin{enumerate}
            \item Of the six materials tested, only three were able to be ignited within below the 750\si{\celsius} maximum temperature of the firebrand surrogate. Two of the three materials for which ignition was not observed were bark. The higher ignition threshold of the bark materials is attributed to higher lignin concentrations. The resistance to ignition present in these materials suggest that using materials high in lignin in locations at risk for firebrand attack in wildfires may reduce ignitions and subsequent losses of structures. It is acknowledged that these results are for flaming ignition only and high lignin materials may pose significant risk if smoldering or the smoldering to flaming transition occurs.
            
            \item For Douglas-fir and pine wood, an increase in wind speed decreased the temperature required for ignition. However, for oak wood, an increase in wind speed inhibited ignition. Increases in ignition propensity with an increase in wind are attributed to the formation of recirculation zones near the ignition source. In the case of oak wood, the formation of recirculation zones inhibits ignition. The root cause of the dichotomy in ignition trends is unclear; however, differences in the flammability of pyrolysis products corresponding to changes in chemical composition are most likely.
            
            \item A direct correlation between the fuel bed composition and the ignition propensity was not observed. This is attributed to the complexity of biomass pyrolysis. However, a local minimum ignition threshold near the composition of Douglas-fir wood is present that warrants further investigation. 
        \end{enumerate}
    The conclusions of this work show that ignition is sensitive to the chemical composition of fuel beds in both quiescent conditions and under wind. An increase in wind was observed to increase ignition probability for some materials; however, this was not universally the case. Materials high in lignin were among the least likely to ignite and more suitable for placement in areas where firebrands are likely to land or accumulate in a WUI environment. Further and more targeted analysis of the differences in pyrolysis products for each material will likely yield further insight. 
    %\cite{MacLean1941}.
    

    
   