%!TEX root = thesis.tex

\renewcommand{\TheTitle}{Influence of Multiple Firebrands on the Ignition of Fuel Beds}
\renewcommand{\TheAuthors}{Derek Bean, David L. Blunck}

\renewcommand{\TheAddress}{
\textbf{Status: In Preparation}
}

\PaperHeader{\TheTitle}{\TheAuthors}{\TheAddress}

\chapter{\TheTitle}
\label{part:manuscript4}

\section{Abstract}
    some text
\section{Introduction}
    The combination of increasing wildfire severity and an increase of homes in the Wildland Urban Interface (WUI) has significantly increased the occurrence of home loss due to wildfires since the turn of the century~\cite{Manzello2013}. Ignition of fuels by firebrands that leads to the loss of a structure is a primary source of home loss~\cite{Koo2010a, Syphard2019, Roberts2021}, and in some fires has been the source of 2/3 of structure ignitions~\cite{Maranghides2011AFires}. Thus, it is important to understand the ignition of fuel beds by firebrands in order to mitigate the risk of ignition~\cite{Manzello2014}. Ideally, the understanding of the ignition process would result in a predictive model that would enable homeowners, firefighters, and other risk management personnel to determine location specific strategies for preventing ignition before and during a fire. Unfortunately, such a model does not currently exist due to critical shortfalls in the current knowledge of fuel bed ignition. 
    
    The ignition of a fuel bed starts within the fire when an ember (e.g., branch, bark, cone etc) is ignited and lofted into the air by wind. The combusting ember is then transported to and lands on a fuel bed near or on a structure. Energy is then transferred to the fuel bed and if the energy is sufficient the fuel bed will undergo pyrolysis and produce flammable gases which may then ignite and begin flaming. These flames may then spread and destroy the structure. Each of these processes, ember generation, transport, and ignition, require additional research before a predictive model is possible. This work focuses on the processes that occur during ignition of the fuel bed. 
    
    An additional risk factor for homes is that sttructure de
\section{Methodology}
    The probability of flaming ignition for fuel beds consisting of Douglas-fir particles was measured for eight configurations of two embers in a wind tunnel. The wind tunnel was operated at either 0.5 \si{\meter\per\second} or 5.8 \si{\meter\per\second} for each test series. The wind speed was measured with a TSI-IFA300 hot wire probe. Figure~\ref{fig:multiHeaterApparatus} shows a representation of the wind tunnel, fuel bed, and the automated lowering device. Table~\ref{tab:multiHeaterConfig} shows the test matrix for each series of tests. For all of the tests the heaters were oriented perpendicular to the flow and the downstream heater was heated. For tests where both heaters were heated the temperatures of both heaters were maintained at the same temperature. 
    
    The combinations of heater spacing and hot or ambient upstream heater were chosen to represent different levels of fluid and thermal interactions between the heaters and the fuel bed. The heater spacing of five diameters was chosen such that minimal interaction between the heaters occurs as previous CFD calculations indicate that the recirculation zone of the upstream heater is approximately five diameters when the heater is in a wind of 5.8 \si{\meter\per\second} and an orientation perpendicular to the flow. Preliminary thermal calculations also indicated that the pyrolysis fronts created by each heater are unlikely to interact within previously observed times to ignition. The heater spacing of one diameter was chosen for a high level of interaction between both the fluid disturbances and thermal fronts of each ember. The one diameter spacing places the downstream heater inside the recirculation zone of the upstream heater under wind and the preliminary heat transfer calculations indicated that the thermal fronts of each heater will merge before the anticipated time to ignition. The chosen heater orientations also represent potential scenarios that may be encountered in a wildfire. The configuration with two heaters both heated is representative of a multi ember attack where the one diameter spacing approximates a firebrand pile and the five diameter spacing approximates two embers falling in close proximity but not within each others region of influence. The configuration with only the downstream heater heated is representative of an ember falling near an object (e.g., twig, rock, or cone) that disturbs the flow and may act as a heat sink for the energy deposited to the ember with the one and five diameter spacing representing an ember falling both within and outside of the region of influence of the ember. The configuration where the upstream heater is not heated and five diameters is also considered a control for comparison to previous single heater ignition results.
    
            \rowcolors{2}{gray!25}{white}
        \begin{table}[hpbt]
            \normalsize
            \caption{Test matrix}
            \centering
            \begin{tabular}{ccccr}
                \rowcolor{gray!50}
               Test Series & Heater Spacing & Upstream Heater & U\textsubscript{bulk} (\si{\meter\per\second})\\
                \hline
                1   & 1 & Ambient & 0.5 \\
                2   & 1 & Ambient & 5.8 \\
                3   & 1 & Hot     & 0.5 \\
                4   & 1 & Hot     & 5.8 \\
                5   & 5 & Ambient & 0.5 \\
                6   & 5 & Ambient & 5.8 \\
                7   & 5 & Hot     & 0.5 \\
                8   & 5 & Hot     & 5.8 \\
            \end{tabular}
            \label{tab:multiHeaterConfig}
        \end{table}
    
    The cartridge heaters used had diameters of 6.4\si{\milli\meter} and lengths of 51\si{\milli\meter}. The heater sizes were chosen to represent large firebrands with a high potential to ignite a fuel bed in a wildfire~\cite{Manzello2007}. The heater temperatures were controlled using a PID temperature controller implemented in LabVIEW. Heater temperatures ranged from 250\si{\celsius} to 750\si{\celsius}. Heater temperatures were measured using a type-K thermocouple attached to the center of each heater opposite the fuel bed.
    Controlling the temperature of the heater provides an advantage over natural burning or smoldering embers and pre-heated particles by removing the temperature variability of the heat source and enabling real time data logging of the heat source temperature. Controlling the temperature of the ember also removes some complexity of calculating the energy imparted to the fuel bed. The heater was lowered onto the fuel bed with an automated lowering device. The lowering device included a load cell and a PID controller to maintain a force equivalent to a two 10\si{\gram} firebrands throughout the experiment. To minimize flow disruptions due to apparatus the heaters were each attached to two 4-40 threaded rods that extended approximately 100\si{\milli\meter} below the load cell. 
    
         \begin{figure}[hbpt]
            \centering
            \resizebox{0.5\columnwidth}{!}{%
                \begin{tikzpicture}
                    \filldraw[pattern=north west lines, pattern color=brown, thick] (2.84, 0)  rectangle (4.34, -.75) node[pos=0.5,rectangle,fill=white] {\scriptsize Fuel Bed};
                    \fill [draw=white,  inner color=red, outer color=white ] (3.43, 0.01) circle (0.15);
                    \fill [draw=white,  inner color=red, outer color=white ] (3.74, 0.01) circle (0.15);
                    \filldraw[draw=black,fill=white, thick] (0, 0)      rectangle (6, 3);
                    \fill[fill=black!50] (3.41, 0) rectangle (3.44, 1.40);
                    \fill[fill=black!50] (3.72, 0) rectangle (3.75, 1.40);
                    \fill[fill=black] (3.55, 1.52) rectangle (4.23, 1.57);
                    \fill[fill=black] (3.55, 1.45) rectangle (3.60, 1.52);
                    \filldraw[draw=black, fill=black!50] (3.3, 1.40) rectangle (3.86, 1.45);
                    \filldraw[draw=black, fill=black!50] (4.09, 1.57) rectangle (4.22, 3);
                   
                    \draw [<-] (3.5, 0.1) -- (3, 0.5) node[left] {\scriptsize Heater};
                    \draw [<-] (3.5, 1.55) -- (3, 1.55) node[left] {\scriptsize Load Cell};
                    \draw [<-] (3.75, 2.5) -- (3, 2.5) node[left] {\scriptsize Lowering Arm};
                    \draw[->]         (0.1, 0.6) -- (0.75, 0.6);
                    \draw[->]         (0.1, 1.2) -- (0.75, 1.2);
                    \node[right] at (-0.75, 1.5) {\scriptsize Inlet};
                    \draw[->]         (0.1, 1.8) -- (0.75, 1.8);
                    \draw[->]         (0.1, 2.4) -- (0.75, 2.4);
                    \draw[draw=black, dashed] (3.336, 0) rectangle (4.54, 0.34);
                    \draw[->]         (4.3, 1) -- (4.0, 0.34);
                    \node[right, align=left] at (4.3, 1) {\scriptsize Computational\\ \scriptsize Domain};
                    \fill[draw=red, fill=red] (3.43, 0.01) circle (0.0635);
                    \fill[draw=red, fill=red] (3.74, 0.01) circle (0.0635);
                \end{tikzpicture}
                }
            \caption{Diagram of the experimental wind tunnel apparatus. Air flows through the wind tunnel from left to right. The dashed region represents the domain subset used for computational efforts.}
            \label{fig:multiHeaterApparatus}
        \end{figure}
    
    The fuel bed material was processed from kiln dried Douglas-fir lumber. The lumber was first planed to generate shavings. The wood shavings were then granulated and screened such that the particles passed through a 2.3\si{\milli\meter} screen but not through a 1.3\si{\milli\meter} screen. The particles were then dried in an oven at 103\si{\celsius} to remove any remaining moisture content. During tests, the fuel was placed in a 140\si{\milli\meter} diameter glass container with a depth of 70\si{\milli\meter} and then inserted into the wind tunnel. The average density of the fuel beds was XX.X\si{\kilo\gram\per\cubic\meter}.
    
    

\section{Results and Discussion}
    some text
\section{Conclusions}
    some text