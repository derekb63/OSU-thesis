%!TEX root = thesis.tex

\renewcommand{\TheTitle}{Influence of Multiple Firebrands on the Ignition of Fuel Beds}
\renewcommand{\TheAuthors}{Derek Bean, David L. Blunck}

\renewcommand{\TheAddress}{
\textbf{Status: In Preparation}
}

\PaperHeader{\TheTitle}{\TheAuthors}{\TheAddress}

\chapter{\TheTitle}
\label{part:manuscript4}

\section{Abstract}
    some text
\section{Introduction}
    Influence of local disturbances both heated and non heated
\section{Methodology}
    The probability of flaming ignition for fuel beds consisting of Douglas-fir particles was measured for various configurations of two embers in a wind tunnel. The wind tunnel was operated at either 0.5 \si{\meter\per\second} or 5.8 \si{\meter\per\second} for each test series. Figure~\ref{fig:multiHeaterApparatus} shows a representation of the wind tunnel, fuel bed, and the automated lowering device. Table~\ref{tab:multiHeaterConfig} shows the test matrix for each series of tests. For all tests the heaters were spaced five diameters apart. This distance was chosen as it corresponded to the approximate size of the recirculation zone of a heater in a wind of 5.8 \si{\meter\per\second} oriented perpendicular to the flow. The wind speed was measured with a TSI-IFA300 hot wire probe. 
    
            \rowcolors{2}{gray!25}{white}
        \begin{table}[hpbt]
            \normalsize
            \caption{Test matrix}
            \centering
            \begin{tabular}{ccccr}
                \rowcolor{gray!50}
               Test Series & Orientation & Heater A & Heater B & U\textsubscript{bulk} (\si{\meter\per\second})\\
                \hline
                1   & Parallel      & Hot  & Cold & 0.5 \\
                2   & Parallel      & Hot  & Cold & 5.8 \\
                3   & Parallel      & Hot  & Hot  & 0.5 \\
                4   & Parallel      & Hot  & Hot  & 5.8 \\
                5   & Perpendicular & Hot  & Cold & 0.5 \\
                6   & Perpendicular & Hot  & Cold & 5.8 \\
                7   & Perpendicular & Cold & Hot  & 0.5 \\
                8   & Perpendicular & Cold & Hot  & 5.8 \\
                9   & Perpendicular & Hot  & Hot  & 0.5 \\
                10  & Perpendicular & Hot  & Hot  & 5.8 
            \end{tabular}
            \label{tab:multiHeaterConfig}
        \end{table}
    
    The cartridge heaters used had diameters of 6.4\si{\milli\meter} and lengths of 51\si{\milli\meter}. The heater sizes were chosen to represent large firebrands with a high potential to ignite a fuel bed in a wildfire~\cite{Manzello2007}. The heater temperatures were controlled using a PID temperature controller implemented in LabVIEW. Heater temperatures ranged from 250\si{\celsius} to 750\si{\celsius}. Heater temperatures were measured using a type-K thermocouple attached to the center of the heater opposite the fuel bed.
    Controlling the temperature of the heater provides an advantage over natural burning or smoldering embers and pre-heated particles by removing the temperature variability of the heat source and enabling real time data logging of the heat source temperature. Controlling the temperature of the ember also removes some complexity of calculating the energy imparted to the fuel bed. The heater was lowered onto the fuel bed with an automated lowering device. The lowering device included a load cell and a PID controller to maintain a force equivalent to a two 10\si{\gram} firebrands throughout the experiment. To minimize flow disruptions due to apparatus the heaters were each attached to two 4-40 threaded rods that extended approximately 100\si{\milli\meter} below the load cell. 
    
         \begin{figure}[hbpt]
            \centering
            \resizebox{0.5\columnwidth}{!}{%
                \begin{tikzpicture}
                    \filldraw[pattern=north west lines, pattern color=brown, thick] (2.84, 0)  rectangle (4.34, -.75) node[pos=0.5,rectangle,fill=white] {\scriptsize Fuel Bed};
                    \fill [draw=white,  inner color=red, outer color=white ] (3.43, 0.01) circle (0.15);
                    \fill [draw=white,  inner color=red, outer color=white ] (3.74, 0.01) circle (0.15);
                    \filldraw[draw=black,fill=white, thick] (0, 0)      rectangle (6, 3);
                    \fill[fill=black!50] (3.41, 0) rectangle (3.44, 1.40);
                    \fill[fill=black!50] (3.72, 0) rectangle (3.75, 1.40);
                    \fill[fill=black] (3.55, 1.52) rectangle (4.23, 1.57);
                    \fill[fill=black] (3.55, 1.45) rectangle (3.60, 1.52);
                    \filldraw[draw=black, fill=black!50] (3.3, 1.40) rectangle (3.86, 1.45);
                    \filldraw[draw=black, fill=black!50] (4.09, 1.57) rectangle (4.22, 3);
                   
                    \draw [<-] (3.5, 0.1) -- (3, 0.5) node[left] {\scriptsize Heater};
                    \draw [<-] (3.5, 1.55) -- (3, 1.55) node[left] {\scriptsize Load Cell};
                    \draw [<-] (3.75, 2.5) -- (3, 2.5) node[left] {\scriptsize Lowering Arm};
                    \draw[->]         (0.1, 0.6) -- (0.75, 0.6);
                    \draw[->]         (0.1, 1.2) -- (0.75, 1.2);
                    \node[right] at (-0.75, 1.5) {\scriptsize Inlet};
                    \draw[->]         (0.1, 1.8) -- (0.75, 1.8);
                    \draw[->]         (0.1, 2.4) -- (0.75, 2.4);
                    \draw[draw=black, dashed] (3.336, 0) rectangle (4.54, 0.34);
                    \draw[->]         (4.3, 1) -- (4.0, 0.34);
                    \node[right, align=left] at (4.3, 1) {\scriptsize Computational\\ \scriptsize Domain};
                    \fill[draw=red, fill=red] (3.43, 0.01) circle (0.0635);
                    \fill[draw=red, fill=red] (3.74, 0.01) circle (0.0635);
                \end{tikzpicture}
                }
            \caption{Diagram of the experimental wind tunnel apparatus. Air flows through the wind tunnel from left to right. The dashed region represents the domain subset used for computational efforts.}
            \label{fig:multiHeaterApparatus}
        \end{figure}
    
    The energy imparted to the fuel bed by the heater(s) was estimated by applying an energy balance to the heater. The calculated heat fluxes to the fuel bed ranged between 5\si{\kilo\watt\per\square\meter} and 50\si{\kilo\watt\per\square\meter} per heater. These values align with those reported for heat fluxes from embers to instrumented substrates~\cite{Tao2020, Hernandez2018}. The power delivered to the heaters were measured with CR9580-10 current sensors and ZMPT101B voltage sensors. Temperature distributions were created from infrared images of the the heater taken with a FLIR SC6700 camera. A black body calibration was performed to correlate photon counts from the camera to temperature and produce a temperature distribution. Temperature profiles were created in the along the axis of the cylindrical heaters. The circumferential temperature of the heater was considered uniform. 
    
    The fuel bed material was processed from kiln dried Douglas-fir lumber. The lumber was first planed to generate shavings. The wood shavings were then granulated and screened such that the particles passed through a 2.3\si{\milli\meter} screen but not through a 1.3\si{\milli\meter} screen. The particles were then dried in an oven at 103\si{\celsius} to remove any remaining moisture content. During tests, the fuel was placed in a 140\si{\milli\meter} diameter glass container with a depth of 70\si{\milli\meter} and then inserted into the wind tunnel. The average density of the fuel beds was XX.X\si{\kilo\gram\per\cubic\meter}.
    
    

\section{Results and Discussion}
    some text
\section{Conclusions}
    some text